	\chapter{Entwicklung eines organischen Objekts} 
	%TODO Einleitung: Idee
	% explizite Bezug auf Ultimaker 2, andere Drucker liefern andere Ergebnisse
	% Besonderheiten eines organischen Objekts

	
	Diese Arbeit bezieht sich h�ufig auf den ultimaker 2, der im Kapitel \ref{chapter:techGr:ultimaker2} vorgestellt wird. In der Arbeit mit einem anderen Drucker k�nnen sich Vorgehensweise und Ergebnis deutlich unterscheiden.
	
	
	
	
	%TODO Ist-Analyse
	\newpage
%	\section{Ist-Analyse}
	
	
	%TODO 
	\section{Konzept}
% --> "Strichm�nnchen"
	
	\section{Entwurf}
	% Arbeit mit Blender
	
	
	\section{Druck des Objekts}
	% Wie w�re es ideal gewesen?, fertiges Objekt 
	% Welche Parameter? --> spezifisch beim organischen Objekt
	% Verlinkung zu Fehlern
	
	\section{Aufgetretene Fehler}
	% Bilder
	% Fehleranalyse, [Ma�nahmen --> technische Grundlagen]
	
	\section{Fazit: Eignung f�r organische Objekte}
	% bezogen auf Ultimaker