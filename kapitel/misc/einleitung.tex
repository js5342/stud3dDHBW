\chapter{Einleitung}

Bei herk�mmlichen Verfahren, um Objekte aus Grundstoffen wie beispielsweise Metallen, Kunststoffen oder Harzen anzufertigen, werden diese oft subtraktiv aus einem gro�en Block des Grundstoffs herausgearbeitet. Dies kann beispielsweise durch Fr�sen erfolgen. 


Neben diesen Verfahren gibt es auch die sogenannten additiven Verfahren, bei denen ein Objekt nach und nach (meist schichtweise) aus dem Grundstoff hergestellt wird. Die Vorteile dieser Verfahren sind unter anderem ein geringerer Materialverbrauch und die Realisierbarkeit komplexerer Formen. 

Einige dieser Verfahren werden in der Industrie bereits verwendet, beispielsweise um medizinische Implantate herzustellen. 
Ein gro�es Anwendungsgebiet liegt im Bereich des Rapid Prototypings. 
Dabei geht es darum, m�glichst schnell Prototypen herzustellen, um die Eignung eines Designs f�r das sp�tere Serienprodukt zu erforschen.
Da hier nur geringe St�ckzahlen ben�tigt werden, f�r die m�glichst keine komplizierten Serienwerkzeuge hergestellt werden sollen, eignen sich additive Verfahren gut f�r diese Anwendung.
In den letzten Jahren wird zunehmend auch der Endkundenmarkt erschlossen. Sogenannte \ac{3D}-Drucker, die in der Regel auf dem Schmelzschichtverfahren basieren, werden inzwischen zu erschwinglichen Preisen angeboten.
\cite[S.5f.]{fastermann2012}
 
Im letzten Jahr hat die \ac{DHBW} einen solchen \ac{3D}-Drucker, den Ultimaker 2, angeschafft. In einer vorangegangenen Studienarbeit wurde dieser in Betrieb genommen und erste Objekte gedruckt. 
Die vorliegende Studienarbeit befasst sich mit dem Designen von Objekten und dem anschlie�enden Drucken mit dem Ultimaker 2.
 
Mithilfe verschiedener Programme sollen ein technisches und ein organisches Objekt erstellt werden. Anschlie�end sollen sie mit dem Ultimaker 2 gedruckt werden. 
Dieser Bericht wird zun�chst einen �berblick �ber verschiedene verbreitete additive Verfahren geben. Anschlie�end werden der Ultimaker 2, die dazugeh�rige Toolchain sowie die verwendeten \ac{CAD}-Programme n�her erl�utert.
Basierend auf diesen Grundlagen wird im Hauptteil des Berichts die Erstellung des technischen und organischen Objekts n�her beschrieben. 