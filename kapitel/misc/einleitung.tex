\chapter{Einleitung}

Bei herk�mmlichen Verfahren, um Objekte aus Grundstoffen wie beispielsweise Metallen, Kunststoffen oder Harzen anzufertigen, werden diese oft subtraktiv aus einem gro�en Block des Grundstoff herausgearbeitet. Dies kann beispielsweise durch Fr�sen erfolgen. 


Neben diesen Verfahren gibt es auch die sogenannten additiven Verfahren, bei denen ein Objekt nach und nach (meist schichtweise) aus dem Grundstoff hergestellt wird. Die Vorteile dieser Verfahren sind unter anderem, dass weniger Material verbraucht wird und auch komplexere Formen relativ einfach realisiert werden k�nnen. 

Einige dieser Verfahren werden in der Industrie bereits verwendet, beispielsweise um medizinische Implantate herzustellen. 
Ein gro�es Anwendungsgebiet liegt im Bereich des Rapid Prototyping. 
Dabei geht es darum, m�glichst schnell Prototypen herzustellen, beispielsweise um die Eignung eines Designs f�r das sp�tere Serienprodukt zu erforschen.
Da hier nur geringe St�ckzahlen ben�tigt werden und m�glichst keine komplizierten Serienwerkzeuge hergestellt werden sollen, eignen sich additive Verfahren gut f�r diese Anwendung.
In den letzten Jahren wird zunehmend auch der Endkundenmarkt erschlossen. Sogenannte \ac{3D}-Drucker, die in der Regel auf dem Extrusionsverfahren basieren, werden inzwischen zu erschwinglichen Preisen angeboten.

 
Im letzten Jahr hat die \ac{DHBW} einen solchen \ac{3D}-Drucker, den Ultimaker 2, angeschafft. In einer vorigen Studienarbeit wurde dieser in Betrieb genommen und erste Objekte gedruckt. 
Diese Studienarbeit befasst sich mit dem Designen von Objekten und dem anschlie�enden Drucken mit dem Ultimaker 2.
 
Das Ziel dieser Studienarbeit ist es, mithilfe verschiedener Programme ein technisches und ein organisches Objekt zu erstellen. Anschlie�end sollen diese mit dem Ultimaker 2 gedruckt werden. 
Die dabei gewonnenen Erkenntnisse sollen auf einer bereits existierenden Website �ber den \ac{3D}-Drucker dokumentiert werden. 
Dieser Bericht wird zun�chst einen �berblick �ber verschiedene verbreitete additive Verfahren geben. Anschlie�end werden der Ultimaker 2, die dazugeh�rige Toolchain sowie die verwendeten \ac{CAD}-Programme n�her erl�utert.
Basierend auf diesen Grundlagen wird dann im Hauptteil des Berichts die Erstellung des technischen und organischen Objekts n�her beschrieben. 