\chapter{Fehler beim Drucken mit dem Ultimaker 2}
	%TODO Unterkap. umbennen
	%TODO Strukturierung
	Dieses Kapitel gibt einen �berblick �ber Fehler, die beim Drucken mit dem Ultimaker 2 auftreten k�nnen. Zudem werden die Fehlerursachen und m�gliche Gegenma�nahmen erl�utert. 

	\section{Olsson}
	.
	\section{Druckplatte nicht haftend}
	Im Idealfall haftet die unterste Schicht des Filaments fest auf der beheizten Druckplatte, um den h�heren Schichten einen guten Halt zu geben. Um das zu erreichen, wird die unterste Schicht meistens langsamer gedruckt als h�here Schichten. 
	
	Manchmal haftet die untere Schicht jedoch nicht richtig auf der Druckplatte. Das kann einerseits dazu f�hren, dass der Druck komplett misslingt, da der Druckkopf die nicht haftenden F�den hinter sich herzieht. Andererseits kann es passieren, dass das Objekt zwar gedruckt wird, sich die Schichten aber mit zunehmender H�he immer weiter gegeneinander verschieben. Wenn die untere Schicht nicht fest auf der Druckplatte haftet, kann es beim folgenden Druck passieren, dass das Objekt vom Druckkopf leicht verschoben wird. Dadurch sitzen die Schichten nicht exakt aufeinander und das Objekt wird schr�g gedruckt.
	
	Dieser Fehler tritt wegen mangelnder Haftung der untersten Filamentschicht auf der Druckplatte auf. 
	
	Als Gegenma�nahme kann die gl�serne Druckplatte mit einer d�nnen Schicht Klebstoff bestrichen werden. Dadurch haftet die erste Filamentschicht wieder besser. 
	Zudem bietet Ultimaker inzwischen eine neue Druckplatte an, die speziell beschichtet ist und dadurch eine bessere Haftung erm�glichen soll. 
	
	\section{Stringing}	
	.
	\section{PTFE}
	.
	\section{zu wenig Material}
	.
	\section{grinding}
	.
	\section{tempsensor}
	.
	% der Schei� mit dem Zugriff auf die Platine
	
	\section{Reibung in Bowdenzug}
	.
	\section{untersch Druckergebnisse je nach Druckposition}
	.
	\section{gebogene Heizplatte}
	.
	\section{keine Haftung der Linien auf anderen Linien}
	.
	\section{schlechte Kalibrierung}
	.
	\section{untersch. Filamente, untersch. Eigenschaften }
	.
	\section{Spannung Feeder}
	.
	\section{Anpressdruck im Feeder}
	.
	\section{Knoten im Filament}
	.
	\section{verschmutzte D�se(au�en und innen)}
	.
	\section{Bollen an Heizblock(total destruction)}
	.
	\section{Falsche Parameter, falsche Supportstruktur}
	.
	\section{Drcuktemperatur}
	.
	\section{Verbranntes Material in D�se}
	.
	\section{mutwillige Zerst�rung durch andere Leute}
	.