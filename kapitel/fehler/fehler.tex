\chapter{Fehler beim Drucken mit dem Ultimaker 2}
	%TODO Unterkap. umbennen
	%TODO Strukturierung
	Dieses Kapitel gibt einen �berblick �ber Fehler, die beim Drucken mit dem Ultimaker 2 auftreten k�nnen. Zudem werden die Fehlerursachen und m�gliche Gegenma�nahmen erl�utert. 

	\section{Olsson}
	.
	\section{Druckplatte nicht haftend}
	\label{chapter:fehler:NichtHaftend}
	Im Idealfall haftet die unterste Schicht des Filaments fest auf der beheizten Druckplatte, um den h�heren Schichten einen guten Halt zu geben. Um das zu erreichen, wird die unterste Schicht meistens langsamer gedruckt als h�here Schichten. 
	
	Manchmal haftet die untere Schicht jedoch nicht richtig auf der Druckplatte. Das kann einerseits dazu f�hren, dass der Druck komplett misslingt, da der Druckkopf die nicht haftenden F�den hinter sich herzieht. Andererseits kann es passieren, dass das Objekt zwar gedruckt wird, sich die Schichten aber mit zunehmender H�he immer weiter gegeneinander verschieben. Wenn die untere Schicht nicht fest auf der Druckplatte haftet, kann es beim folgenden Druck passieren, dass das Objekt vom Druckkopf leicht verschoben wird. Dadurch sitzen die Schichten nicht exakt aufeinander und das Objekt wird schr�g gedruckt.
	
	Dieser Fehler tritt wegen mangelnder Haftung der untersten Filamentschicht auf der Druckplatte auf. 
	
	Als Gegenma�nahme kann die gl�serne Druckplatte mit einer d�nnen Schicht Klebstoff bestrichen werden. Dadurch haftet die erste Filamentschicht wieder besser. 
	Zudem bietet Ultimaker inzwischen eine neue Druckplatte an, die speziell beschichtet ist und dadurch eine bessere Haftung erm�glichen soll. 
	
	\section{Stringing}	
	.
	\section{PTFE}
	%TODO: Bilder
	In \ref{chapter:techGr:ultimaker2} wird der Aufbau des Extruders genauer beschrieben. Die D�se ist zwar aus Metall gefertigt, aber das Filament gelangt durch eine \ac{PTFE}-Kopplung in die D�se. Dadurch, dass die Kopplung in der N�he des Heizblocks sitzt, wird sie beim Drucken ebenfalls erhitzt. Das f�hrt einerseits dazu, dass die Kopplung sich relativ schnell verformt und andererseits "verbrennt".
	
	Die Verformung f�hrt dazu, dass das Filament nicht mehr ungehindert durch die Kopplung in die D�se gelangen kann. Dadurch wird teilweise zu wenig Filament gef�rdert und der Druck misslingt. 
	
	Deshalb sollte die \ac{PTFE}-Kopplung regelm��ig ausgetauscht werden.
	
	\section{zu wenig Material}
	\label{chapter:fehler:zuWenigMaterial}
	.
	\section{grinding}
	.
	\section{tempsensor}
	.
	% der Schei� mit dem Zugriff auf die Platine
	
	\section{Reibung in Bowdenzug}
	\label{chapter:fehler:ReibungBowden}
	.
	\section{untersch Druckergebnisse je nach Druckposition}
	Abh�ngig von der Position auf der Druckplatte, an der dasselbe Objekt zum Drucken platziert wird, ergeben sich teilweise stark unterschiedliche Ergebnisse. An manchen Positionen gelingt der Druck gut, w�hrend an anderen Fehler auftreten. Je nach Position haftet das Objekt unterschiedlich stark an der Druckplatte. Auch die Menge an Filament, die gef�rdert werden kann, h�ngt von der Position ab.
	
	Die unterschiedlich starke Haftung kann mehrere Ursachen haben: Da die Glasplatte nicht perfekt auf der Heizplatte aufliegt \ref{chapter:fehler:gebogeneHeizplatte}, kann es sein, dass verschiedene Positionen unterschiedlich gut erhitzt werden. An manchen Positionen haftet das Objekt somit schlechter, da diese nicht so gut beheizt werden.
	Eine weitere Ursache kann sein, dass der Klebstoff, der die Haftung des Filaments auf der Druckplatte verbessern soll \ref{chapter:fehler:NichtHaftend}, nicht gleichm��ig aufgetragen wurde.
	
	Dass an manchen Positionen zu wenig Filament gef�rdert wird \ref{chapter:fehler:zuWenigMaterial} kann daran liegen, dass der Widerstand im Bowdenzug variiert, je nachdem, wie stark der Bowdenzug gekr�mmt ist \ref{chapter:fehler:ReibungBowden}.
	.
	\section{gebogene Heizplatte}
	\label{chapter:fehler:gebogeneHeizplatte}
	.
	\section{keine Haftung der Linien auf anderen Linien}
	.
	\section{schlechte Kalibrierung}
	.
	\section{untersch. Filamente, untersch. Eigenschaften }
	.
	\section{Spannung Feeder}
	.
	\section{Anpressdruck im Feeder}
	.
	\section{Knoten im Filament}
	.
	\section{verschmutzte D�se(au�en und innen)}
	.
	\section{Bollen an Heizblock(total destruction)}
	.
	\section{Falsche Parameter, falsche Supportstruktur}
	.
	\section{Drcuktemperatur}
	.
	\section{Verbranntes Material in D�se}
	.
	\section{mutwillige Zerst�rung durch andere Leute}
	.