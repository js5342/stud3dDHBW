In diesem Kapitel wird betrachtet, wie stabil die Objekte sind, die mit den in \ref{drucktechniken} vorgestellten Verfahren hergestellt werden. 

\subsection{Schmelzschichtverfahren} 
Im Extrusionsverfahren hergestellte Objekte sind mechanisch und thermisch belastbarer als Objekte, die mit Stereolithografie erstellt wurden. \cite[S.10]{gebhardt2004grundlagen} \\
Es kann allerdings vorkommen, dass manche Bereiche eines Objekts weniger stabil sind als andere Bereiche. \cite{hagl2014_3DKompendiumTechnologien}



\subsection{Stereolithographie}
Objekte, die mit Stereolithografie hergestellt werden, sind mechanisch und thermisch weniger belastbar als Objekte, die beispielsweise mit Lasersintern oder im Extrusionsverfahren hergestellt wurden. \cite[S.4]{gebhardt2004grundlagen}\\
Zudem ist unklar, ob diese Objekte stabil genug f�r einen Dauereinsatz sind. \cite{hagl2014_3DKompendiumTechnologien}


\subsection{Elektronenstrahlschmelzen}  %TODO

\subsection{Lasersintering}
Mit diesem Verfahren k�nnen sowohl Metalle als auch Kunststoffe verarbeitet werden. Die dabei produzierten Teile k�nnen �hnliche Eigenschaften wie mit herk�mmlichen Methoden produzierte Teile aufweisen.    \\
Allerdings sind gesinterte Teile por�s, was jedoch bei manchen Anwendungsf�llen erw�nscht sein kann. Die Porosit�t kann abgeschw�cht werden, indem das Pulver mit anderen Materialien versetzt wird. \cite{hagl2014_3DKompendiumTechnologien}


\subsection{Binder-Verfahren}  
Bei diesem Verfahren ist es schwer, eine Aussage �ber die Eigenschaften des entstehenden Materials zu treffen, da es sich aus Pulver, Bindemittel und dem zur Nachbehandlung genutzten Harz zusammensetzt.\\
Sofern als Pulver kein Metall, sondern Gips oder St�rke verwendet werden, h�lt das Objekt keinen starken Belastungen stand. \cite[S.12]{gebhardt2004grundlagen}


\subsection{Schicht-Laminat-Verfahren}
Da bei diesem Verfahren die Objekte aus verschiedenen Schichten aufgebaut sind, die miteinander verpresst oder verklebt wurden, weisen diese Objekte richtungsabh�ngig verschiedene Eigenschaften auf.
\cite[S.8]{gebhardt2004grundlagen}\\
Zudem wird bei diesem Verfahren oft Papier als Material verwendet. Naturgem�� ist Papier nicht lange haltbar. \cite[S.35]{hagl2014_3DKompendiumTechnologien}
