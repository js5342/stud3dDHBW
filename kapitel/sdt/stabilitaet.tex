In diesem Kapitel soll betrachtet werden, wie stabil die Teile sind, die mit den vorgestellten Technologien hergestellt werden. Insbesondere der Vergleich mit herk�mmlich produzierten Teilen soll dabei betrachtet werden.

\subsection{Extrusionsverfahren}
%gebh: + mech+therm bel h�her als stereol, -raue oberfl, geringere details

\subsection{Stereolithographie}
% fl�ssig: ungenau
% nachbehandlung
% langfristige Stabilit�t "Sorgenpunkt"
% + gro�e objekte, hohe pr�zision, oberfl�chenveredlung{\tiny 

%Gebhardt: -mech + therm belastbarkeit schlcehter als lasersintern/extrusion

\subsection{Elektronenstrahlschmelzen}

\subsection{Lasersintering}
Dieses Verfahren kann sowohl Metall als auch Kunststoff verarbeiten. Die dabei produzierten Teile reichen qualitativ an herk�mmlich produzierte Teile heran.   \\
Allerdings sind gesinterte Teile por�s, was bei manchen Anwendungsf�llen erw�nscht sein kann. Die Porosit�t kann abgeschw�cht werden, indem das Pulver mit anderen Materialien versetzt wird.
\cite{hagl2014_3DKompendium}

% \subsection{Binder-Verfahren}
\subsection{Schicht-Laminat-Verfahren}
%gebh: -richtungsabh. eigenschaften, geringere genauigkeit
