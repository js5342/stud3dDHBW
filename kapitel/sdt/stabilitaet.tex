In diesem Kapitel wird betrachtet, wie stabil die Objekte sind, die mit den vorgestellten Verfahren hergestellt werden. 

\subsection{Extrusionsverfahren} %TODO
%gebh: + mech+therm bel h�her als stereol, -raue oberfl, geringere details

\subsection{Stereolithographie}
Objekte, die mit Stereolithografie hergestellt werden, sind mechanisch und thermisch weniger belastbar als Objekte, die beispielsweise mit Lasersintern oder im Extrusionsverfahren hergestellt wurden. \cite{gebhardt2004grundlagen}\\
Zudem ist unklar, ob diese Objekte stabil genug f�r einen Dauereinsatz sind. \cite{hagl2014_3DKompendiumTechnologien}
% fl�ssig: ungenau
% nachbehandlung
% langfristige Stabilit�t "Sorgenpunkt"
% + gro�e objekte, hohe pr�zision, oberfl�chenveredlung 

%Gebhardt: -mech + therm belastbarkeit schlcehter als lasersintern/extrusion

\subsection{Elektronenstrahlschmelzen}  %TODO

\subsection{Lasersintering}
Dieses Verfahren kann sowohl Metall als auch Kunststoff verarbeiten. Die dabei produzierten Teile k�nnen �hnliche Eigenschaften wie herk�mmlich produzierte Teile aufweisen.    \\
Allerdings sind gesinterte Teile por�s, was bei manchen Anwendungsf�llen erw�nscht sein kann. Die Porosit�t kann abgeschw�cht werden, indem das Pulver mit anderen Materialien versetzt wird. \cite{hagl2014_3DKompendiumTechnologien}


% \subsection{Binder-Verfahren}  %TODO
\subsection{Schicht-Laminat-Verfahren}
Da bei diesem Verfahren die Objekte aus verschiedenen Schichten aufgebaut sind, die miteinander verpresst oder verklebt wurden, weisen diese Objekte richtungsabh�ngig verschiedene Eigenschaften auf.
\cite{gebhardt2004grundlagen}
%gebh: -richtungsabh. eigenschaften, geringere genauigkeit
