\section{Materialien}
	Die verschiedenen Druckverfahren erfordern unterschiedliche Grundstoffe f�r das Drucken. Dieser Abschnitt stellt verschiedene Materialien vor.
	
	\subsection{Metalle und deren Legierungen}
		Pulver verschiedener Metalle und Legierungen lassen sich sintern. Bleche k�nnen im Schicht-Lamitat-Verfahren \ref{drucktechniken:schichtLaminatVerfahren} zu einem festen Objekt geformt werden. Im Allgemeinen lassen sich aus Metall per 3D- Druck mechanisch und thermisch belastbare Prototypen erstellen.
	\subsection{Monomere}
		Im Stereolithographie-Verfahren \ref{drucktechniken:stereolithographie} werden Monomere selektiv polymerisiert. Monomere sind Molek�le, die mit gleichartigen Molek�len zu gr��eren Molek�len verschmelzen k�nnen. Durch Polymerisation entsteht ein fester K�rper aus langkettigen Molek�len. 
		\cite{Abts2014Polymere}
		
	\subsection{Thermoplaste}
		Bereits polymerisierte Kunststoffe 
	