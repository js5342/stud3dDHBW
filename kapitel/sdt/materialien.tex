\section{Materialien}
	Die verschiedenen Druckverfahren erfordern unterschiedliche Grundstoffe f�r das Drucken. Dieser Abschnitt stellt verschiedene Materialien vor.
	
	\subsection{Metalle und deren Legierungen}
		Pulver verschiedener Metalle und Legierungen lassen sich sintern. Bleche k�nnen im Schicht-Lamitat-Verfahren \ref{drucktechniken:schichtLaminatVerfahren} zu einem festen Objekt geformt werden. Im Allgemeinen lassen sich aus Metall per 3D- Druck mechanisch und thermisch belastbare Prototypen erstellen.
	\subsection{Monomere}
		Im Stereolithographie-Verfahren \ref{drucktechniken:stereolithographie} werden Monomere selektiv polymerisiert. Monomere sind Molek�le, die mit gleichartigen Molek�len zu gr��eren Molek�len verschmelzen k�nnen. Durch Polymerisation entsteht ein fester K�rper aus langkettigen Molek�len. 
		\cite{Abts2014Polymere}
		
	\subsection{Thermoplaste}
		Bereits polymerisierte Kunststoffe unterscheiden sich in der Reaktion auf hohe Temperaturen. Eine dieser Gruppen von Polymeren sind die Thermoplaste. Diese Kunststoffe verfl�ssigen sich bei Temperatureinwirkung und erstarren beim anschlie�enden Ausk�hlen in einer neuen Form. Wenn die Temperatur zu hoch ist, verschmoren Thermoplaste. Daher muss beim Drucken eine Temperatur gefunden werden, die das Material in eine verwendbare Liquidit�t versetzt, allerdings das Material nicht besch�digt. Bekannte Thermoplaste sind \ac{PLA} und \ac{ABS}. Ebenfalls l�sst sich Schokolade drucken, die ebenfalls zu den Thermoplasten zugeordnet werden kann. Beim Drucken mit \ac{ABS} entstehen giftige D�mpfe. Die Menge an ausgeschiedenem Gas ist bei korrekter Druckertemperatur-Einstellung unbedenklich, allerdings sollte der Druckraum trotzdem gut ventiliert werden.
		
		Im Gegensatz zu den Thermoplasten verfl�ssigen sich Duroplaste nicht; sie verschmoren direkt, wenn die Temperatur zu hoch wird. Dadurch eignen sie sich nicht f�r das Extrusionsverfahren.  