\section{Materialien}
	Die verschiedenen Druckverfahren erfordern unterschiedliche Grundstoffe f�r das Drucken. In diesem Abschnitt werden verschiedene Materialien vorgestellt.
	
	\subsection{Metalle und deren Legierungen}
		Pulver verschiedener Metalle und Legierungen lassen sich sintern. Bleche k�nnen im Schicht-Laminat-Verfahren \ref{drucktechniken:schichtLaminatVerfahren} zu einem festen Objekt geformt werden. Im Allgemeinen lassen sich aus Metall mit 3D- Druckverfahren mechanisch und thermisch belastbare Prototypen erstellen.
	\subsection{Monomere}
		Im Stereolithographie-Verfahren \ref{drucktechniken:stereolithographie} werden Monomere selektiv polymerisiert. Monomere sind Molek�le, die mit gleichartigen Molek�len zu gr��eren Molek�len verschmelzen k�nnen. Durch Polymerisation entsteht ein fester K�rper aus langkettigen Molek�len. 
		\cite{Abts2014Polymere}
		
	\subsection{Thermoplaste}
		Bereits polymerisierte Kunststoffe unterscheiden sich in ihrer Reaktion auf hohe Temperaturen. Eine Gruppen von Polymeren sind die Thermoplaste.
		
		Diese Kunststoffe verfl�ssigen sich bei Temperatureinwirkung und erstarren beim anschlie�enden Ausk�hlen in einer neuen Form. Wenn die Temperatur zu hoch ist, verschmoren Thermoplaste. Daher muss beim Drucken eine Temperatur gefunden werden, die hoch genug ist, um das Material so zu verfl�ssigen, dass es gut gedruckt werden kann. Die Temperatur darf allerdings nicht zu hoch sein, damit das Material nicht besch�digt wird. Bekannte Thermoplaste sind \ac{PLA} und \ac{ABS}. 
		%Ebenfalls l�sst sich Schokolade drucken, die ebenfalls zu den Thermoplasten zugeordnet werden kann.
		 
		Beim Drucken mit \ac{ABS} entstehen giftige D�mpfe. Die Menge an ausgeschiedenem Gas ist unbedenklich, wenn die Druckertemperatur korrekt eingestellt wird, allerdings sollte der Druckraum trotzdem gut ventiliert werden.
		
		Im Gegensatz zu den Thermoplasten verfl�ssigen sich Duroplaste nicht; sie verschmoren direkt, wenn die Temperatur zu hoch wird. Dadurch eignet sich diese Gruppe von Polymeren nicht f�r das Extrusionsverfahren.  \cite[S.510ff.]{christen1974}
		S.510ff