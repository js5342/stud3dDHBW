Dreidimensionale Objekte k�nnen mit verschiedenen Verfahren gedruckt werden. Viele werden bereits seit Langem eingesetzt. Die Technologie war folglich schon verf�gbar. In den letzten Jahre erreichen die Drucker den Endkundenmarkt. Grund daf�r sind eine Vielzahl von g�nstigen Druckern, die au dem Markt erh�ltlich sind. Im Folgenden sind verschiedene Druckverfahren erl�utert.

\subsection{Extrusionsverfahren}%TODO 
Das verfl�ssigte Material wird durch eine D�se auf eine Druckfl�che gepresst. Dort h�rtet es aus. Durch Bewegen des Druckkopfes �ber die Druckfl�che l�sst sich Schicht f�r Schicht ein Objekt auftragen.



\subsection{Stereolithographie}

Das dreidimensionale �quivalent zum Rasterdruck baut Objekte aus Schichten von Rasterpunkten auf.

Manche organischen Verbindungen k�nnen mittels \ac{UV}- Licht polymerisiert werden. Dadurch wird aus einem fl�ssigen Grundstoff ein fester K�rper. Dieser Vorgang wird als Photopolymerisation bezeichnet und dient als Grundlage f�r verschiedene \ac{3D}- Druckverfahren. 

F�r jede Schicht kann eine Photomaske erzeugt werden. Der fl�ssige Grundstoff wird durch die Maske hindurch von einer \ac{UV}- Quelle angestrahlt. Dadurch h�rtet eine Schicht des Grundmaterials entsprechend der Photomaske aus. Die erste Schicht wird auf eine Bodenplatte gedruckt. Diese wird nach Abschlie�en jeder Schicht weiter in die Fl�ssigkeit abgesenkt. Dadurch wird die n�chste Schicht auf die darunterliegende gedruckt.


Alternativ zur Photomaske mit gleichm��iger Quelle kann auch ein \ac{UV}- Laser verwendet werden, der �ber Spiegel an die auszuh�rtenden Rasterpunkte gelenkt wird.

Das erzeugte Objekt wird im Druckverfahren nicht vollst�ndig ausgeh�rtet. Daher muss es im Anschluss mit \ac{UV}- Licht nachbehandelt werden.
\cite[S.3]{gebhardt2004grundlagen}

\subsection{Sintern}
 Sintern beschreibt den Prozess des Verdichtens pulverf�rmiger Ausgangsstoffe zu einem festen Material. Hierzu kann das Material �ber den Schmelzpunkt erhitzt werden oder durch hohe Dr�cke dazu verleitet werden, dass sich die Oberfl�chen der einzelnen Pulverk�rner verbinden.
 F�r den \ac{3D}- Druck relevant sind Verfahren, die ein selektives Verbinden der K�rner erm�glichen.
 \cite[Bd.20, S.7037]{meyers2006}
 
 %	\subsubsection{Elektronenstrahlschmelzen}
 Ein Verfahren, das dies erm�glicht, ist das Elektronenstrahlschmelzen. Hierbei wird schichtweise das Pulver des Ausgangsmaterials selektiv mit einem Elektronenstrahl geschmolzen. Nach Fertigstellung einer Schicht wird eine weitere Schicht Pulver aufgetragen, die erneut selektiv geschmolzen werden kann. Dadurch k�nnen \ac{3D} Objekte erzeugt werden. Momentan sind Objekte aus mehreren Titanlegierungen mit diesem Verfahren m�glich. Zudem wird an der Eignung von Stahl, verschiedenen Metallen und deren Legierungen geforscht.
 \cite{IFAMElektronenstrahl}
 
 %\subsubsection{Lasersintern}
 Alternativ zum Elektronenstrahl kann auch ein Laser zum Verschwei�en des Pulvers eingesetzt werden. Mit diesem Verfahren k�nnen auch Kunststoffe verarbeitet werden. 	
\cite{lasersintern}
	
\subsection{Binder-Verfahren}
Im Binderverfahren wird ein Bindemittel in ein pulverf�rmiges Ausgangsmaterial eingespritzt. Durch selektives Einbringen des Binders k�nnen die gew�nschten Strukturen erzeugt werden.
\cite[S.11]{gebhardt2004grundlagen}