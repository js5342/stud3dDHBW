Dreidimensionale Objekte können mit verschiedenen Verfahren gedruckt werden. Viele werden bereits seit Langem eingesetzt. Die Technologie war folglich schon verfügbar. In den letzten Jahre erreichen die Drucker den Endkundenmarkt. Grund dafür sind eine Vielzahl von günstigen Druckern, die au dem Markt erhältlich sind. Im Folgenden sind verschiedene Druckverfahren erläutert.

\subsection{Extrusion, Schmelzschichtung}
Das verflüssigte Material wird durch eine Düse auf eine Druckfläche gepresst. Dort härtet es aus. Durch Bewegen des Druckkopfes über die Druckfläche lässt sich Schicht für Schicht ein Objekt auftragen.



\subsection{Stereolithographie}

Das dreidimensionale Äquivalent zum Rasterdruck baut Objekte aus Schichten von Rasterpunkten auf.

Manche organischen Verbindungen können mittels \ac{UV}- Licht polymerisiert werden. Dadurch wird aus einem flüssigen Grundstoff ein fester Körper. Dieser Vorgang wird als Photopolymerisation bezeichnet und dient als Grundlage für verschiedene \ac{3D}- Druckverfahren. 

Für jede Schicht kann eine Photomaske erzeugt werden. Der flüssige Grundstoff wird durch die Maske hindurch von einer \ac{UV}- Quelle angestrahlt. Dadurch härtet eine Schicht des Grundmaterials entsprechend der Photomaske aus. Die erste Schicht wird auf eine Bodenplatte gedruckt. Diese wird nach Abschließen jeder Schicht weiter in die Flüssigkeit abgesenkt. Dadurch wird die nächste Schicht auf die darunterliegende gedruckt.


Alternativ zur Photomaske mit gleichmäßiger Quelle kann auch ein \ac{UV}- Laser verwendet werden, der über Spiegel an die auszuhärtenden Rasterpunkte gelenkt wird.


\subsection{Sintern}

	\subsubsection{Elektronenstrahlschmelzen}
	\subsubsection{Lasersintern}

\subsection{Binder-Verfahren}